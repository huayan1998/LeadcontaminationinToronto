% Options for packages loaded elsewhere
\PassOptionsToPackage{unicode}{hyperref}
\PassOptionsToPackage{hyphens}{url}
\PassOptionsToPackage{dvipsnames,svgnames,x11names}{xcolor}
%
\documentclass[
  letterpaper,
  DIV=11,
  numbers=noendperiod]{scrartcl}

\usepackage{amsmath,amssymb}
\usepackage{iftex}
\ifPDFTeX
  \usepackage[T1]{fontenc}
  \usepackage[utf8]{inputenc}
  \usepackage{textcomp} % provide euro and other symbols
\else % if luatex or xetex
  \usepackage{unicode-math}
  \defaultfontfeatures{Scale=MatchLowercase}
  \defaultfontfeatures[\rmfamily]{Ligatures=TeX,Scale=1}
\fi
\usepackage{lmodern}
\ifPDFTeX\else  
    % xetex/luatex font selection
\fi
% Use upquote if available, for straight quotes in verbatim environments
\IfFileExists{upquote.sty}{\usepackage{upquote}}{}
\IfFileExists{microtype.sty}{% use microtype if available
  \usepackage[]{microtype}
  \UseMicrotypeSet[protrusion]{basicmath} % disable protrusion for tt fonts
}{}
\makeatletter
\@ifundefined{KOMAClassName}{% if non-KOMA class
  \IfFileExists{parskip.sty}{%
    \usepackage{parskip}
  }{% else
    \setlength{\parindent}{0pt}
    \setlength{\parskip}{6pt plus 2pt minus 1pt}}
}{% if KOMA class
  \KOMAoptions{parskip=half}}
\makeatother
\usepackage{xcolor}
\setlength{\emergencystretch}{3em} % prevent overfull lines
\setcounter{secnumdepth}{5}
% Make \paragraph and \subparagraph free-standing
\ifx\paragraph\undefined\else
  \let\oldparagraph\paragraph
  \renewcommand{\paragraph}[1]{\oldparagraph{#1}\mbox{}}
\fi
\ifx\subparagraph\undefined\else
  \let\oldsubparagraph\subparagraph
  \renewcommand{\subparagraph}[1]{\oldsubparagraph{#1}\mbox{}}
\fi


\providecommand{\tightlist}{%
  \setlength{\itemsep}{0pt}\setlength{\parskip}{0pt}}\usepackage{longtable,booktabs,array}
\usepackage{calc} % for calculating minipage widths
% Correct order of tables after \paragraph or \subparagraph
\usepackage{etoolbox}
\makeatletter
\patchcmd\longtable{\par}{\if@noskipsec\mbox{}\fi\par}{}{}
\makeatother
% Allow footnotes in longtable head/foot
\IfFileExists{footnotehyper.sty}{\usepackage{footnotehyper}}{\usepackage{footnote}}
\makesavenoteenv{longtable}
\usepackage{graphicx}
\makeatletter
\def\maxwidth{\ifdim\Gin@nat@width>\linewidth\linewidth\else\Gin@nat@width\fi}
\def\maxheight{\ifdim\Gin@nat@height>\textheight\textheight\else\Gin@nat@height\fi}
\makeatother
% Scale images if necessary, so that they will not overflow the page
% margins by default, and it is still possible to overwrite the defaults
% using explicit options in \includegraphics[width, height, ...]{}
\setkeys{Gin}{width=\maxwidth,height=\maxheight,keepaspectratio}
% Set default figure placement to htbp
\makeatletter
\def\fps@figure{htbp}
\makeatother

\KOMAoption{captions}{tableheading}
\makeatletter
\makeatother
\makeatletter
\makeatother
\makeatletter
\@ifpackageloaded{caption}{}{\usepackage{caption}}
\AtBeginDocument{%
\ifdefined\contentsname
  \renewcommand*\contentsname{Table of contents}
\else
  \newcommand\contentsname{Table of contents}
\fi
\ifdefined\listfigurename
  \renewcommand*\listfigurename{List of Figures}
\else
  \newcommand\listfigurename{List of Figures}
\fi
\ifdefined\listtablename
  \renewcommand*\listtablename{List of Tables}
\else
  \newcommand\listtablename{List of Tables}
\fi
\ifdefined\figurename
  \renewcommand*\figurename{Figure}
\else
  \newcommand\figurename{Figure}
\fi
\ifdefined\tablename
  \renewcommand*\tablename{Table}
\else
  \newcommand\tablename{Table}
\fi
}
\@ifpackageloaded{float}{}{\usepackage{float}}
\floatstyle{ruled}
\@ifundefined{c@chapter}{\newfloat{codelisting}{h}{lop}}{\newfloat{codelisting}{h}{lop}[chapter]}
\floatname{codelisting}{Listing}
\newcommand*\listoflistings{\listof{codelisting}{List of Listings}}
\makeatother
\makeatletter
\@ifpackageloaded{caption}{}{\usepackage{caption}}
\@ifpackageloaded{subcaption}{}{\usepackage{subcaption}}
\makeatother
\makeatletter
\@ifpackageloaded{tcolorbox}{}{\usepackage[skins,breakable]{tcolorbox}}
\makeatother
\makeatletter
\@ifundefined{shadecolor}{\definecolor{shadecolor}{rgb}{.97, .97, .97}}
\makeatother
\makeatletter
\makeatother
\makeatletter
\makeatother
\ifLuaTeX
  \usepackage{selnolig}  % disable illegal ligatures
\fi
\IfFileExists{bookmark.sty}{\usepackage{bookmark}}{\usepackage{hyperref}}
\IfFileExists{xurl.sty}{\usepackage{xurl}}{} % add URL line breaks if available
\urlstyle{same} % disable monospaced font for URLs
\hypersetup{
  pdftitle={Lead Contamination in Toronto's Drinking Water: A Comprehensive Analysis},
  pdfauthor={Your Name},
  colorlinks=true,
  linkcolor={blue},
  filecolor={Maroon},
  citecolor={Blue},
  urlcolor={Blue},
  pdfcreator={LaTeX via pandoc}}

\title{Lead Contamination in Toronto's Drinking Water: A Comprehensive
Analysis\thanks{Code and data are available at: LINK.}}
\usepackage{etoolbox}
\makeatletter
\providecommand{\subtitle}[1]{% add subtitle to \maketitle
  \apptocmd{\@title}{\par {\large #1 \par}}{}{}
}
\makeatother
\subtitle{My subtitle if needed}
\author{First author \and Another author}
\date{January 29, 2024}

\begin{document}
\maketitle
\begin{abstract}
First sentence. Second sentence. Third sentence. Fourth sentence.
\end{abstract}
\ifdefined\Shaded\renewenvironment{Shaded}{\begin{tcolorbox}[frame hidden, interior hidden, borderline west={3pt}{0pt}{shadecolor}, boxrule=0pt, sharp corners, enhanced, breakable]}{\end{tcolorbox}}\fi

\hypertarget{abstract}{%
\section{Abstract}\label{abstract}}

This paper investigates the issue of lead contamination in Toronto's
drinking water supply. Through an analysis of historical data and water
quality measures, we aim to understand the extent of lead exposure and
its potential health impacts. The top-level finding reveals that despite
significant efforts to mitigate lead contamination, challenges persist
in ensuring the safety of tap water, particularly for vulnerable
populations. This paper sheds light on the ongoing concerns related to
lead in drinking water.

\hypertarget{introduction}{%
\section{Introduction}\label{introduction}}

Lead contamination in drinking water has been a growing concern in
various parts of the world. In Toronto, water quality is a top priority,
and the City has implemented several measures to address lead exposure.
However, this paper explores the sources of lead contamination, its
impact on health, and the effectiveness of mitigation strategies. We
also discuss the importance of transparency and public awareness in
addressing this issue.

This paper is organized as follows: In the following sections, we will
delve into the data sources used for our analysis and provide a
comprehensive overview of the dataset. We will then discuss the
findings, implications, and potential recommendations for addressing
lead contamination in Toronto's drinking water.

\hypertarget{data}{%
\section{Data}\label{data}}

\hypertarget{data-source}{%
\subsection{Data Source}\label{data-source}}

Our analysis is based on data collected by the City of Toronto regarding
water quality and lead levels in the drinking water supply. The dataset
includes information on lead service pipes, water testing results, and
corrosion control measures implemented by the City.

\hypertarget{data-context}{%
\subsection{Data Context}\label{data-context}}

Lead contamination in drinking water is a critical public health
concern. Even low levels of lead exposure can have harmful effects,
particularly on children and pregnant women. This dataset captures the
historical trends in lead levels, water quality measures, and the
implementation of corrosion control measures.

\hypertarget{table-1-summary-of-data}{%
\subsubsection{Table 1: Summary of Data}\label{table-1-summary-of-data}}

\begin{longtable}[]{@{}
  >{\raggedright\arraybackslash}p{(\columnwidth - 2\tabcolsep) * \real{0.4028}}
  >{\raggedright\arraybackslash}p{(\columnwidth - 2\tabcolsep) * \real{0.5972}}@{}}
\toprule\noalign{}
\begin{minipage}[b]{\linewidth}\raggedright
Variable
\end{minipage} & \begin{minipage}[b]{\linewidth}\raggedright
Description
\end{minipage} \\
\midrule\noalign{}
\endhead
\bottomrule\noalign{}
\endlastfoot
Lead Service Pipes & Information on the presence of lead pipes \\
Water Testing Results & Historical lead levels in tap water \\
Corrosion Control Measures & Implementation of corrosion control \\
\end{longtable}

\hypertarget{figure-1-trends-in-lead-levels}{%
\subsubsection{Figure 1: Trends in Lead
Levels}\label{figure-1-trends-in-lead-levels}}

{[}Insert your ggplot2 graph here{]}

\hypertarget{figure-2-corrosion-control-measures-over-time}{%
\subsubsection{Figure 2: Corrosion Control Measures Over
Time}\label{figure-2-corrosion-control-measures-over-time}}

{[}Insert your ggplot2 graph here{]}

\hypertarget{references}{%
\section{References}\label{references}}

{[}Include your BibTeX references here{]}

\hypertarget{acknowledgments}{%
\section{Acknowledgments}\label{acknowledgments}}

We acknowledge the support of the open data community in Toronto and the
City of Toronto for providing access to the water quality dataset. The
GitHub repository for this project can be found
\href{https://github.com/yourusername/your-repo}{here}.

\newpage

\hypertarget{references-1}{%
\section{References}\label{references-1}}



\end{document}
